\documentclass{article}

\usepackage[greek]{babel}

\usepackage[a4paper, margin=1.15in]{geometry}

\usepackage{graphicx}
\graphicspath{{./screenshots}}

\usepackage[export]{adjustbox}
\newcommand{\includescreenshot}[1]{\includegraphics[max width=\textwidth]{#1}}

\usepackage{fontspec}
\setmainfont{DejaVuSansM Nerd Font}
\setsansfont{DejaVuSansM Nerd Font}

\title{
	Portal Maze\\
	\large Απαλλακτική Εργασία Γραφικών\\
	Μέρος Α
}
\author{Αλκίνοος Αλυσσανδράκης 1072752}
\date{}

\begin{document}

\maketitle

Το πρώτο μέρος της εργασίας ζητούσε τα εξής:

\begin{enumerate}
	\item Δημιουργία κύβων με διάφορα υλικά οι οποίοι θα χρησιμοποιηθούν για την κατασκευή
		ενός λαβύρινθου
	\item Δημιουργία ενός αλγόριθμου που θα κατασκευάζει τυχαίους λαβύρινθους χρησιμοποιώνατς
		τους κύβους
	\item Φωτισμός και σκίαση της σκηνής
	\item Δημιουργία ενός avatar που θα αναπαριστά τον παίκτη. Αρχικά η κάμερα θα κινείται
		με βάση τους περιορισμούς του παίκτη (να μην αιωρείται και να μην μπαίνει στους
		τοίχους του λαβύρινθου)
	\item Δημιουργία λειτουργίας στην οποία η κάμερα θα ελευθερώνεται από τον παίκτη και
		θα μπορεί να κινείται ελεύθερα μέσα στη σκηνή
\end{enumerate}

\noindent
Στην παρούσα φάση η εργασία βρίσκεται στην εξής κατάσταση:

Έχουν δημιουργηθεί κύβοι με ένα υλικό με τους οποίους κατασκευάζεται ένας λαβύρινθος (ο ίδος
σε κάθε εκτέλεση του προγράμματος). Η προσθήκη περισσότερων υλικών θα είναι απλή
διαδικασία η οποία θα γίνει σε δεύτερη φάση αφού ολοκληρωθεί η διαδικασία κατασκευής του
λαβύρινθου.

Ο φωτισμός της σκηνής γίνεται με τεχνικές Physically Based Rendering (PBR).
Από τη διαδικασία του φωτισμού λείπει η δυνατότητα για normal και displacement mappinng,
οι οποίες θα προστεθούν στην πορεία. Προς το παρόν δεν υπάρχει κάποια τεχνική σκίασης και
εξετάζεται το ενδεχόμενο να μην προστεθούν σκιες, τουλάχιστον μέχρι να ολοκληρωθεί ένα
μεγάλο κομμάτι από την υπόλοιπη εργασία.

Έχει προστεθεί ένα avatar για τον παίκτη (το μοντέλο της suzanne) το οποίο κινείται μαζί
με την κάμερα στη λειτουργία first person κατά την οποία κάμερα και avatar κινούνται στις
διαστάσεις x,z,pitch, yaw. Με τα πλήκτρα WASD γίνεται η κίνηση μπροστά, αριστερά,
πίσω και δεξιά αντίστοιχα. Με το πλήκτρο Space γίνεται η εναλλαγή από τη λειτουργία
first person στη λειτουργία free camera, στην οποία η κάμερα είναι ελεύθερη να κινηθεί
στις διαστάσεις x,y,z,pitch,yaw. Ακόμα δεν έχει προστεθεί κάποιου είδους περιορισμός ώστε
ο παίκτης να μην μπορεί να κινηθεί μέσα από τους τοίχους του λαβύρινθου, όμως υπάρχουν
σχέδια να προστεθεί σύστημα collision detection για αυτό.

Σε γενικές γραμμές έχει ολοκληρωθεί ένα 75\% του πρώτου μέρους της εργασίας, ενώ πολύ
σύντομα θα προστεθεί η τυχαία δημιουργία του λαβύρινθου και collision detection για τον
παικτη (το οποίο θα χρησιμεύσει και όταν προστεθούν τα portals) που είναι πιο σημαντικά
και πιθανώς να βελτιωθεί το shading.

\begin{figure}
	\includescreenshot{1.png}
	\caption{First person view}
\end{figure}
\begin{figure}
	\includescreenshot{2.png}
	\caption{Free camera view}
\end{figure}

\end{document}
